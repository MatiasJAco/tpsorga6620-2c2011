\documentclass[a4paper,11pt]{article}

\usepackage[spanish]{babel}
\usepackage{multirow}
\usepackage{verbatim}
\usepackage{moreverb}
\usepackage{graphicx}

%Idioma
\usepackage[latin1]{inputenc}
%Letra arial
\usepackage{helvet}
%\renewcommand\familydefault{\sfdefault}
%Para incluir codigo fuente
\usepackage{listings}
%Para tener encabezados y pie de pagina personalizados
\usepackage{fancyhdr}
%Para poner eprafe en las imagenes
\usepackage[hang,bf]{caption2}

%%------------Gráficos------------
%%Paquete de gráficos
%\newif\ifpdf
%\ifx\pdfoutput\undefined
%	\pdffalse
%\else
%	\pdfoutput=1
%	\pdftrue
%\fi
%
%\ifpdf
%	\usepackage[pdftex]{graphicx}
%	\pdfcompresslevel=9
%% 	\pdfcompresslevel=0
%	\usepackage{pdfpages}
%\else
%	\usepackage[dvips]{graphicx}
%\fi



% Titulo del Trabajo Practico.
\title{ Trabajo Pr\'actico 2: MIPS Datapath   \\
        \Large{ 66.20 Organizaci\'on de las Computadoras } }


% Informaci\'on sobre los autores.
\author{Nicol\'as Calvo, \textit{Padr\'on Nro. 78.914}           	\\
            \texttt{ nicolas.g.calvo@gmail.com }      			\\
            Celeste Maldonado, \textit{Padr\'on Nro. 85.630}              	\\
            \texttt{ maldonado.celeste@gmail.com }                             \\
            Matias Acosta, \textit{Padr\'on Nro. 88.590}                \\
            \texttt{ matiasja@gmail.com }                     		\\
            \LARGE{}         						\\
            \LARGE{}         						\\
            \LARGE{}         						\\
            \Large{2do. Cuatrimestre de 2011}         	\\                       
            \texttt{}         						\\
            \Large{Facultad de Ingenier\'\i{}a, Universidad de Buenos Aires}            \\
       }
\date{}



\begin{document}

\begin{figure}
\centering
\includegraphics[width=100pt]{logofiuba.jpg}
\end{figure}


\maketitle
\thispagestyle{empty}   % quita el numero en la primer pagina
%\renewcommand{\labelenumi}{\alph{enumi}.}


\newpage
% Declaro el indice

\tableofcontents
\newpage

\setcounter{page}{1}

\section{Introducci\'on}


\paragraph{}


\section{Desarrollo}

\begin{enumerate}

\item	
\item	
\item 
\item


\paragraph{Ejecuci\'on del programa sin forwarding}


\paragraph{}
Se ejecutan 129 ciclos y 64 instrucciones , con 2 instrucciones en el  pipeline al finalizar.

\paragraph{}
\begin{center}
$CPI = 129 ciclos / 64 instrucciones = 2.02$
\end{center}

\paragraph{}
Se contaron un total de 64 stalls, divididos en las siguientes categorías:

\begin{itemize}
\item 10 stalls de control (7.75% de todos los ciclos)
\item 2  stalls correspondientes a la instruccion trap (1.55% de todos los ciclos)
\item 52 stalls RAW (Read After Write) (40.31% de todos los ciclos).
\end{itemize}

\paragraph{Ejecuci\'on del programa con forwarding:}


\paragraph{}
Se ejecutan 98 ciclos y 64 instrucciones , con 2 instrucciones en el  pipeline al finalizar.

\paragraph{}
\begin{center}
$CPI = 98 ciclos / 64 instrucciones = 1,53$
\end{center}

\paragraph{}
Se logr\'o un $SpeedUp = 1.32$

\paragraph{} 
Se contaron un total de 33 stalls, divididos en las siguientes categorías:

\begin{itemize}
\item 10 stalls de control (10.20% de todos los ciclos)
\item 2  stalls correspondientes a la instruccion trap (2.04\% de todos los ciclos)
\item 21 stalls RAW (Read After Write) (21.43\% de todos los ciclos), de los cuales los 21 corresponden a stalls de branch, es decir, que dichos branches tienen por argumentos registros escritos en la instruccion que los precede.
\end{itemize}

\paragraph{} 
Para el codigo presentado Branch Delay Slot no podría usarse para lograr una mejora significativa en tiempo de ejecuci\'on debido a las dependencias de los branches respecto a los argumentos de las instrucciones que se ejecutan antes y despues de ellos.

\paragraph{} 
Por ejemplo, para el segmento:

\begin{center}
\begin{verbatim}
		andi r3,r2,#1 
		bnez r3,Modulo 
		add r1,r1,r2 
Modulo:	sgt r8,r1,r6
\end{verbatim}
\end{center}

\paragraph{} 
La instruccion andi se usa en el branch, por lo que no puede moverse al delay slot y sgt requiere de un registro que es modificado en caso de no tomar el branch, por lo que mover esta instruccion al delay slot modificaría la logica del programa.


\item

\paragraph{} 
Se reorden\'o el codigo moviendo a la instrucci\'on 

\begin{verbatim}
andi r3,r2,#1 
\end{verbatim}


de la siguiente forma:

\paragraph{}
Codigo original:

\begin{center}
\begin{verbatim}
	sge r8,r2,r5
	bnez r8,Fin 
	andi r3,r2,#1 
	bnez r3,Modulo 
	add r1,r1,r2
\end{verbatim}
\end{center}

\paragraph{} 
Codigo reordenado:

\begin{center}
\begin{verbatim}
	sge r8,r2,r5 
	andi r3,r2,#1 
	bnez r8,Fin 	 
	bnez r3,Modulo 
	add r1,r1,r2
\end{verbatim}
\end{center}

\paragraph{}
De esta forma se espera reducir los ciclos de stall RAW para que bnez r3, Modulo tenga disponible el valor del registro r3. Algo similar sucedería para bnez r8,Fin respecto a sge r8,r2,r5.

\paragraph{}
Se obtuvieron los siguientes resultados:

\paragraph{Ejecucion sin forwarding:}

\paragraph{}
Se ejecutan 108 ciclos y 64 instrucciones , con 2 instrucciones en el  pipeline al finalizar.

\begin{center}
$CPI = 108 ciclos / 64 instrucciones = 1.69 $
\end{center}

\paragraph{}
Se contaron un total de 43 stalls, divididos en las siguientes categorías:

\begin{itemize}
 \item 10 stalls de control (9.26% de todos los ciclos)
 \item  2  stalls correspondientes a la instruccion trap (1.85% de todos los ciclos)
 \item  31 stalls RAW (Read After Write) (28.70% de todos los ciclos)
\end{itemize}

\paragraph{}
SpeedUp respecto al codigo $sin reordenamiento = 1.2$

\paragraph{Ejecuci\'on con forwarding:}

\paragraph{}9
Se ejecutan 84 ciclos y 64 instrucciones , con 2 instrucciones en el  pipeline al finalizar.

\begin{center}
$CPI = 84 ciclos / 64 instrucciones = 1.31$
\end{center}

\paragraph{}
SpeedUp respecto al codigo$ sin reordenamiento = 1.17$

\paragraph{} 
Se contaron un total de 19 stalls, divididos en las siguientes categor'ias:

\begin{itemize}
 \item 10 stalls de control (11.09% de todos los ciclos)
 \item  stalls correspondientes a la instruccion trap (2.38% de todos los ciclos)
 \item stalls RAW (Read After Write) (8.33% de todos los ciclos), de los cuales los 7 corresponden a stalls de branch, es decir, que dichos branches tienen por argumentos registros escritos en la instruccion que los precede.
\end{itemize}


\item




\end{enumerate}

\section{Conclusi\'on}



\end{document}

